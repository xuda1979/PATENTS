% Options for packages loaded elsewhere
\PassOptionsToPackage{unicode}{hyperref}
\PassOptionsToPackage{hyphens}{url}
\documentclass[
]{article}
\usepackage{xcolor}
\usepackage{amsmath,amssymb}
\setcounter{secnumdepth}{-\maxdimen} % remove section numbering
\usepackage{iftex}
\ifPDFTeX
  \usepackage[T1]{fontenc}
  \usepackage[utf8]{inputenc}
  \usepackage{textcomp} % provide euro and other symbols
\else % if luatex or xetex
  \usepackage{unicode-math} % this also loads fontspec
  \defaultfontfeatures{Scale=MatchLowercase}
  \defaultfontfeatures[\rmfamily]{Ligatures=TeX,Scale=1}
\fi
\usepackage{lmodern}
\ifPDFTeX\else
  % xetex/luatex font selection
\fi
% Use upquote if available, for straight quotes in verbatim environments
\IfFileExists{upquote.sty}{\usepackage{upquote}}{}
\IfFileExists{microtype.sty}{% use microtype if available
  \usepackage[]{microtype}
  \UseMicrotypeSet[protrusion]{basicmath} % disable protrusion for tt fonts
}{}
\makeatletter
\@ifundefined{KOMAClassName}{% if non-KOMA class
  \IfFileExists{parskip.sty}{%
    \usepackage{parskip}
  }{% else
    \setlength{\parindent}{0pt}
    \setlength{\parskip}{6pt plus 2pt minus 1pt}}
}{% if KOMA class
  \KOMAoptions{parskip=half}}
\makeatother
\usepackage{longtable,booktabs,array}
\newcounter{none} % for unnumbered tables
\usepackage{calc} % for calculating minipage widths
% Correct order of tables after \paragraph or \subparagraph
\usepackage{etoolbox}
\makeatletter
\patchcmd\longtable{\par}{\if@noskipsec\mbox{}\fi\par}{}{}
\makeatother
% Allow footnotes in longtable head/foot
\IfFileExists{footnotehyper.sty}{\usepackage{footnotehyper}}{\usepackage{footnote}}
\makesavenoteenv{longtable}
\setlength{\emergencystretch}{3em} % prevent overfull lines
\providecommand{\tightlist}{%
  \setlength{\itemsep}{0pt}\setlength{\parskip}{0pt}}
\usepackage{bookmark}
\IfFileExists{xurl.sty}{\usepackage{xurl}}{} % add URL line breaks if available
\urlstyle{same}
\hypersetup{
  hidelinks,
  pdfcreator={LaTeX via pandoc}}

\author{}
\date{}

\begin{document}

\section{Patent Abstract}\label{patent-abstract}

\subsection{Quantum-Secure Threshold Signature System and Method Based
on Lattice-Based Falcon
Algorithm}\label{quantum-secure-threshold-signature-system-and-method-based-on-lattice-based-falcon-algorithm}

\begin{center}\rule{0.5\linewidth}{0.5pt}\end{center}

\subsubsection{Abstract}\label{abstract}

\textbf{Technical Field}: The present invention relates to blockchain
technology, Post-Quantum Cryptography (PQC), and Multi-Party Computation
(MPC).

\textbf{Technical Problem}: Existing cross-chain bridges using elliptic
curve threshold signatures are vulnerable to quantum computing attacks.
Furthermore, distributed implementation of the Falcon algorithm faces
discrete Gaussian sampling challenges and O(n) communication complexity.

\textbf{Technical Solution}: A quantum-secure threshold signature system
comprising: a distributed key generation module for generating
secret-shared NTRU trapdoor portions among multiple nodes via secure
multi-party computation; an arithmetic-shared NTT computation module
utilizing NTT linearity for zero-communication distributed polynomial
operations; a collaborative rejection sampling module reducing
communication rounds from O(n) to O(1) through a pre-check commitment
mechanism; and a signature aggregation module outputting NIST
Falcon-compliant digital signatures.

\textbf{Technical Effects}: Compared to existing Dilithium threshold
schemes, signature length is reduced by approximately 3.6 times
(\textasciitilde666 bytes vs \textasciitilde2420 bytes), blockchain gas
costs are reduced by approximately 72\%, dynamic node management is
supported, while maintaining quantum-resistant security properties.

\begin{center}\rule{0.5\linewidth}{0.5pt}\end{center}

\subsubsection{Keywords}\label{keywords}

Post-quantum cryptography; Falcon signature; Threshold signature;
Multi-party secure computation; NTRU lattice; Cross-chain bridge;
Discrete Gaussian sampling

\begin{center}\rule{0.5\linewidth}{0.5pt}\end{center}

\subsubsection{Brief Description of
Drawings}\label{brief-description-of-drawings}

\textbf{Figure 1}: Overall system architecture diagram showing
interaction flow between source chain, threshold signature system, and
target chain

\textbf{Figure 2}: Collaborative rejection sampling flowchart showing
the three-phase protocol of commitment, pre-check, and reveal

\textbf{Figure 3}: Dynamic node management diagram including three
scenarios: node addition, revocation, and offline recovery

\begin{center}\rule{0.5\linewidth}{0.5pt}\end{center}

\subsubsection{IPC Classifications}\label{ipc-classifications}

\begin{itemize}
\tightlist
\item
  H04L 9/32 --- Digital signatures
\item
  H04L 9/30 --- Public key cryptosystems
\item
  H04L 9/08 --- Key distribution
\item
  G06F 21/64 --- Integrity protection
\end{itemize}

\begin{center}\rule{0.5\linewidth}{0.5pt}\end{center}

\subsubsection{Technical Effects
Summary}\label{technical-effects-summary}

{\def\LTcaptype{none} % do not increment counter
\begin{longtable}[]{@{}
  >{\raggedright\arraybackslash}p{(\linewidth - 6\tabcolsep) * \real{0.2609}}
  >{\raggedright\arraybackslash}p{(\linewidth - 6\tabcolsep) * \real{0.2319}}
  >{\raggedright\arraybackslash}p{(\linewidth - 6\tabcolsep) * \real{0.3188}}
  >{\raggedright\arraybackslash}p{(\linewidth - 6\tabcolsep) * \real{0.1884}}@{}}
\toprule\noalign{}
\begin{minipage}[b]{\linewidth}\raggedright
Technical Metric
\end{minipage} & \begin{minipage}[b]{\linewidth}\raggedright
This Invention
\end{minipage} & \begin{minipage}[b]{\linewidth}\raggedright
Prior Art (Dilithium)
\end{minipage} & \begin{minipage}[b]{\linewidth}\raggedright
Improvement
\end{minipage} \\
\midrule\noalign{}
\endhead
\bottomrule\noalign{}
\endlastfoot
Signature Length & \textasciitilde666 bytes & \textasciitilde2420 bytes
& 3.6x smaller \\
Communication Rounds & O(1) & O(n) & Significantly reduced \\
Gas Cost & \textasciitilde50,000 & \textasciitilde180,000 & 72\%
savings \\
Quantum-Safe & ✓ & ✓ & Equivalent \\
Dynamic Nodes & ✓ & Limited & Enhanced \\
\end{longtable}
}

\begin{center}\rule{0.5\linewidth}{0.5pt}\end{center}

\subsubsection{Document Checklist}\label{document-checklist}

\begin{enumerate}
\def\labelenumi{\arabic{enumi}.}
\tightlist
\item
  Patent Application (Chinese) - \texttt{专利申请书\_中文.md}
\item
  Patent Application (English) - \texttt{patent\_draft.md}
\item
  Technical Specification - \texttt{technical\_specification.md}
\item
  Claims (English) - \texttt{claims\_EN.md}
\item
  Abstract (Chinese) - \texttt{摘要\_中文.md}
\item
  Abstract (English) - This document
\item
  Drawing Specifications - \texttt{drawings\_specification.md}
\item
  Prior Art Search Report - \texttt{prior\_art\_report.md}
\item
  Experimental Data - \texttt{experimental\_data.md}
\end{enumerate}

\begin{center}\rule{0.5\linewidth}{0.5pt}\end{center}

\subsubsection{Word Count}\label{word-count}

Abstract body word count: \textasciitilde150 words (compliant with USPTO
150-word limit)

\begin{center}\rule{0.5\linewidth}{0.5pt}\end{center}

\subsubsection{Priority and Filing
Strategy}\label{priority-and-filing-strategy}

\textbf{Recommended Filing Sequence:}

\begin{enumerate}
\def\labelenumi{\arabic{enumi}.}
\tightlist
\item
  \textbf{Initial Filing (China)}: Submit to CNIPA to establish priority
  date
\item
  \textbf{PCT Application}: File within 12 months claiming China
  priority
\item
  \textbf{National Phase}: Enter US, EU, and other jurisdictions within
  30/31 months
\end{enumerate}

\textbf{Priority Date}: {[}To be determined upon filing{]}

\begin{center}\rule{0.5\linewidth}{0.5pt}\end{center}

\emph{This abstract is prepared for international patent filing through
the PCT route. The Chinese application serves as the priority document.}

\end{document}
