% ============================================
% CLAIMS / 权利要求书
% ============================================
\newpage
\section{权利要求书}

\subsection{独立权利要求}

\subsubsection{权利要求 1(系统权利要求)}

一种面向量子计算网络的抗量子攻击节点协同认证系统,包括通过网络连接的多个量子计算节点,每个节点包括处理器和存储加密指令及密钥分片的存储器,所述系统包括:

\begin{enumerate}[label=\alph*), leftmargin=2em]
    \item \textbf{可验证分布式密钥生成模块},配置用于利用安全多方计算在 $n$ 个节点之间生成满足 $fG - gF = q$ 的 NTRU 陷门秘密共享分片,其中:
    \begin{itemize}[label=$\bullet$]
        \item 每个节点 $P_i$ 在其本地存储器中持有陷门分片 $([f]_i, [g]_i, [F]_i, [G]_i)$,使得 $\sum_{i=1}^{n}[f]_i = f$,且 $g, F, G$ 同理;
        \item 任何少于门限数量 $t$ 的节点联盟除了公钥 $h = g \cdot f^{-1} \mod q$ 之外,无法获得关于完整私钥的任何信息;
        \item 通过加密承诺和零知识证明确保分片生成的正确性,从而实现可验证性;
    \end{itemize}

    \item \textbf{算术共享变换域计算模块},配置用于以分布式方式执行 Falcon 签名生成所需的多项式运算,其中:
    \begin{itemize}[label=$\bullet$]
        \item 每个节点处理器在本地对其私钥分片 $[f]_i$ 计算数论变换(NTT);
        \item 利用 NTT 的线性性质,使得 $\text{NTT}(\sum_{i=1}^{n} [f]_i) = \sum_{i=1}^{n} \text{NTT}([f]_i)$;
        \item 变换域多项式乘法通过分片上的逐点运算执行:$[\text{NTT}(f \cdot g)]_i = \text{NTT}([f]_i) \odot \text{NTT}(g)$;
        \item 在任何计算阶段都不需要重构私钥多项式;
    \end{itemize}

    \item \textbf{具有方差保持聚合功能的分布式高斯采样模块},配置用于生成分布正确的样本,其中:
    \begin{itemize}[label=$\bullet$]
        \item 每个节点 $P_i$ 从缩放的离散高斯分布 $[z]_i \leftarrow D_{\sigma/\sqrt{n}, R}$ 中采样,其中 $\sigma$ 是目标 Falcon 参数;
        \item 根据高斯卷积定理,聚合值 $z = \sum_{i=1}^{n}[z]_i$ 服从目标分布 $D_{\sigma, R}$;
        \item 个体样本的统计独立性确保了每个节点贡献的隐私性;
    \end{itemize}

    \item \textbf{协同拒绝采样模块},配置用于执行具有隐私保护的分布式接受测试,包括:
    \begin{itemize}[label=$\bullet$]
        \item 本地掩码生成子模块,用于在每个节点生成随机掩码 $m_i$;
        \item 承诺子模块,用于通过网络接口计算并广播本地签名分片的加密承诺 $C_i = H(m_i \| [s]_i)$;
        \item 安全范数计算子模块,利用 Beaver 三元组预处理在常数轮次内计算全局签名范数:
        $$\|s\|^2 = \sum_{i}\|[s]_i\|^2 + 2\sum_{i<j}\langle[s]_i, [s]_j\rangle$$
        \item 分布式抛币子模块,用于以正比于 $\exp(-\|s\|^2/(2\sigma^2))$ 的概率进行协同接受决策;
        \item 其中在线通信复杂度从 $O(n)$ 轮降低到确切的 6 轮常数轮次;
    \end{itemize}

    \item \textbf{签名聚合与验证模块},配置用于聚合来自参与节点的签名分量,并输出标准 Falcon 格式的数字签名 $\sigma = (r, \text{Compress}(s_2))$,该签名可使用未经修改的标准 Falcon 验证算法进行验证。
\end{enumerate}

\subsubsection{权利要求 2(方法权利要求)}

一种面向量子计算网络的抗量子攻击节点协同认证方法,由分布在量子计算网络中的多个节点处理器执行,包括以下步骤:

\begin{description}[leftmargin=2em, style=nextline]
    \item[S1) 可验证分布式密钥生成:] $n$ 个签名节点通过安全多方计算协议协同生成 NTRU 陷门,包括:
    \begin{itemize}[label=$\bullet$]
        \item 每个节点 $P_i$ 从缩放的离散高斯分布中采样本地多项式分片 $[f]_i, [g]_i \leftarrow D_{\sigma/\sqrt{n}, R}$;
        \item 计算承诺 $C_i = \text{Commit}([f]_i, r_i)$ 并生成正确采样的零知识证明;
        \item 跨节点验证所有承诺和证明;
        \item 执行 MPC 扩展欧几里得(MPC-Extended-GCD)协议,在秘密共享多项式上求解 $fG - gF = q$;
        \item 将陷门分片 $([f]_i, [g]_i, [F]_i, [G]_i)$ 分发给每个节点,并发布公共公钥 $h = g \cdot f^{-1} \mod q$;
    \end{itemize}

    \item[S2) 离线预处理:] 使用后量子安全的不经意传输扩展生成满足 $\sum_i [c]_i = (\sum_i [a]_i) \cdot (\sum_i [b]_i)$ 的 Beaver 乘法三元组 $\{([a]_i, [b]_i, [c]_i)\}$,并通过牺牲协议验证,确保正确性概率为 $1 - 2^{-40}$;

    \item[S3) 消息预处理:] 接收待签名的消息 $M$,通过承诺-揭示机制采样分布式随机盐值 $r$,使用 SHAKE-256 计算加密哈希 $c = H(r \| M)$,并将哈希值映射到 $R_q$ 中的目标多项式;

    \item[S4) 本地签名分片计算:] 每个节点 $P_i$ 执行:
    \begin{itemize}[label=$\bullet$]
        \item 计算本地陷门贡献 $[t]_i = (\text{NTT}^{-1}(\text{NTT}([F]_i) \odot \text{NTT}(c)), \text{NTT}^{-1}(\text{NTT}([G]_i) \odot \text{NTT}(c)))$;
        \item 采样具有缩放参数的本地高斯噪声 $[z]_i \leftarrow D_{\sigma/\sqrt{n}, R}^2$;
        \item 计算掩码签名分片 $[s]_i = [t]_i + [z]_i$;
        \item 使用随机掩码 $m_i$ 生成承诺 $C_i = H(m_i \| [s]_i)$;
    \end{itemize}

    \item[S5) 具有常数轮次安全聚合的协同拒绝采样:] 包括:
    \begin{itemize}[label=$\bullet$]
        \item 第 1 轮:广播承诺 $\{C_i\}$ 以将各方绑定到分片;
        \item 第 2-3 轮:使用 Beaver 乘法协议计算所有 $i < j$ 的交叉项内积 $\langle[s]_i, [s]_j\rangle$;
        \item 第 4 轮:聚合并不经意揭示全局范数 $\|s\|^2$;
        \item 第 5 轮:执行分布式抛币,以概率 $p = M^{-1} \cdot \exp(-\langle s, c \rangle / \sigma^2)$ 确定接受;
        \item 如果被拒绝,返回步骤 S4 进行重采样;
    \end{itemize}

    \item[S6) 签名聚合:] 在接受后:
    \begin{itemize}[label=$\bullet$]
        \item 第 6 轮:每个节点揭示签名分片 $[s]_i$;
        \item 针对承诺验证揭示的分片;
        \item 聚合分量 $(s_1, s_2) = \sum_{i=1}^{n}[s]_i$;
        \item 根据 Falcon 压缩算法压缩签名分量 $s_2$;
        \item 输出标准 Falcon 格式的最终签名 $\sigma = (r, \text{Compress}(s_2))$。
    \end{itemize}
\end{description}

\subsubsection{权利要求 3(独立方法权利要求 - 分布式密钥生成)}

一种用于量子安全门限签名系统的 NTRU 陷门可验证分布式生成方法,由 $n$ 个计算节点执行,包括:

\begin{enumerate}[label=\alph*), leftmargin=2em]
    \item \textbf{分布式多项式采样}:每个节点 $P_i$ 独立地从缩放的离散高斯分布 $D_{\sigma/\sqrt{n}, R}$ 中采样本地多项式分片 $[f]_i, [g]_i$,使得分片之和 $f = \sum [f]_i$ 和 $g = \sum [g]_i$ 服从目标分布 $D_{\sigma, R}$;

    \item \textbf{可验证承诺}:每个节点广播其本地分片的加密承诺,并提供格式良好的零知识证明,以确保没有节点采样熵不足的分布;

    \item \textbf{安全逆计算}:节点利用安全多方计算(MPC)协议协同计算环 $R_q$ 中多项式 $f$ 的共享逆,以生成公钥 $h = g \cdot f^{-1} \mod q$;

    \item \textbf{分布式陷门补全}:节点执行基于 MPC 的扩展欧几里得算法(XGCD),协同查找满足 NTRU 方程的秘密共享多项式 $[F]_i$ 和 $[G]_i$:
    $$f \cdot \left(\sum [G]_i\right) - g \cdot \left(\sum [F]_i\right) = q$$
    且没有任何节点获知完整多项式 $f, g, F, \text{或 } G$;

    \item \textbf{分片输出}:每个节点将结果陷门分片元组 $([f]_i, [g]_i, [F]_i, [G]_i)$ 存储在安全存储器中,以支持后续的门限签名操作。
\end{enumerate}

\newpage
\subsection{从属权利要求}

\subsubsection{依赖于权利要求 1(系统)的权利要求}

\textbf{权利要求 4.} 根据权利要求 1 所述的系统,其中所述算术共享变换域计算模块利用中国剩余定理同构 $R_q \cong \mathbb{Z}_q^n$ 执行多项式乘法:
$$\text{NTT}(f \cdot g) = \text{NTT}(f) \odot \text{NTT}(g)$$
从而实现变换操作本身的零节点间通信的每系数并行计算。

\vspace{0.5cm}

\textbf{权利要求 5.} 根据权利要求 1 所述的系统,其中所述分布式高斯采样模块实现方差保持聚合,满足:
$$\text{Var}(\sum [z]_i) = \sum \text{Var}([z]_i) = n \cdot (\sigma/\sqrt{n})^2 = \sigma^2$$
确保聚合噪声分布与标准 Falcon 签名算法完全不可区分。

\vspace{0.5cm}

\textbf{权利要求 6.} 根据权利要求 1 所述的系统,其中所述协同拒绝采样模块利用 Beaver 三元组实现安全内积计算,包括:
\begin{itemize}[label=$\bullet$]
    \item 离线阶段预生成满足 $c = a \cdot b$ 的乘法三元组 $([a]_i, [b]_i, [c]_i)$;
    \item 在线阶段通过开放掩码差值计算交叉项,无需暴露原始分片。
\end{itemize}

\vspace{0.5cm}

\textbf{权利要求 7.} 根据权利要求 1 所述的系统,还包括动态节点管理模块,配置用于:
\begin{itemize}[label=$\bullet$]
    \item 在不更改公钥的情况下,通过秘密共享协议向新节点分发私钥分片;
    \item 通过主动秘密共享更新剩余节点的私钥分片以撤销节点;
    \item 当检测到节点离线时,通过门限恢复协同重构该节点的私钥分片。
\end{itemize}

\vspace{0.5cm}

\textbf{权利要求 8.} 根据权利要求 1 所述的系统,其中每个硬件节点包括可信执行环境(TEE),所述 TEE 配置用于:
\begin{itemize}[label=$\bullet$]
    \item 在 TEE 内存中以明文形式保护密钥分片;
    \item 在加入签名组之前提供远程认证报告;
    \item 使用 TEE 密封密钥将密钥分片持久化到存储器。
\end{itemize}

\vspace{0.5cm}

\subsubsection{依赖于权利要求 2(方法)的权利要求}

\textbf{权利要求 9.} 根据权利要求 2 所述的方法,其中步骤 S2 中的不经意传输扩展使用后量子安全的 Kyber-KEM 原语实例化基础 OT。

\vspace{0.5cm}

\textbf{权利要求 10.} 根据权利要求 2 所述的方法,其中步骤 S5 的协同拒绝采样的接受概率为:
$$p = \frac{1}{M} \cdot \exp\left(-\frac{\langle s, c \rangle}{\sigma^2}\right) \approx 0.65$$
使得平均重试次数约为 1.53 次。

\vspace{0.5cm}

\textbf{权利要求 11.} 根据权利要求 2 所述的方法,还包括容错处理步骤:
\begin{itemize}[label=$\bullet$]
    \item 为每个协议阶段配置可配置的超时时间;
    \item 当参与方超时时,将其从当前签名尝试中排除;
    \item 如果剩余参与方数量 $|S \setminus \{P_i\}| \geq t$,使用缩减后的集合继续。
\end{itemize}

\vspace{0.5cm}

\textbf{权利要求 12.} 根据权利要求 2 所述的方法,其中步骤 S6 的签名输出为标准 Falcon-512 格式,签名长度约为 666 字节,可在以太坊虚拟机上以约 50,000 Gas 进行验证。

\newpage
