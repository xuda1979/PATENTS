% ============================================
% TITLE PAGE / 封面
% ============================================

\begin{titlepage}
    \centering
    \vspace*{1cm}
    
    {\LARGE\textbf{中华人民共和国}}\par
    {\LARGE\textbf{发明专利申请}}\par
    \vspace{0.3cm}
    {\large Patent Application for Invention}\par
    {\large People's Republic of China}\par
    
    \vspace{1.5cm}
    
    \rule{\textwidth}{1pt}
    
    \vspace{1cm}
    
    {\huge\textbf{发明名称}}\par
    \vspace{0.5cm}
    {\LARGE\textbf{面向量子计算网络的}}\par
    {\LARGE\textbf{抗量子攻击节点协同认证系统及方法}}\par
    \vspace{0.3cm}
    {\large Anti-Quantum Attack Node Collaborative Authentication}\par
    {\large System and Method for Quantum Computing Networks}\par
    
    \vspace{1.5cm}
    
    \rule{\textwidth}{1pt}
    
    \vspace{1cm}
    
    \begin{tabular}{rl}
        \textbf{申请人:} & 中国移动通信有限公司研究院 \\[0.4cm]
        \textbf{发明人:} & 许达 (Xu Da) \\[0.4cm]
        \textbf{申请日期:} & \today \\[0.4cm]
        \textbf{IPC分类号:} & H04L 9/32; H04L 9/30; H04L 9/08; G06F 21/64 \\[0.4cm]
    \end{tabular}
    
    \vfill
    
    \begin{center}
    \begin{tabular}{|p{14cm}|}
    \hline
    \textbf{技术领域:} 量子计算技术、后量子密码学、多方安全计算 \\
    \textbf{主要创新点:} \\
    • 算术共享NTT协议实现零通信分布式多项式运算 \\
    • 基于Rényi散度的分布式高斯采样 \\
    • 6轮恒定通信的协同纠偏采样 \\
    • 面向量子计算网络的高效节点认证验证机制 \\
    \hline
    \end{tabular}
    \end{center}
    
    \vspace{1cm}
    
    {\large 中国国家知识产权局}\par
    {\large China National Intellectual Property Administration (CNIPA)}
    
\end{titlepage}
