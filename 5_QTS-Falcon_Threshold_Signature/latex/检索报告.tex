\documentclass[12pt,a4paper]{article}
\usepackage[UTF8]{ctex}
\usepackage{geometry}
\usepackage{amsmath,amssymb}
\usepackage{graphicx}
\usepackage{booktabs}
\usepackage{longtable}
\usepackage{array}
\usepackage{enumitem}
\usepackage{hyperref}
\usepackage{fancyhdr}
\usepackage{xcolor}
\usepackage{colortbl}
\usepackage{setspace}

\geometry{left=2.5cm,right=2.5cm,top=2.5cm,bottom=2.5cm}
\setlength{\headheight}{15pt}
\setstretch{1.5}

% 去掉页眉中的“现有技术检索报告”
\pagestyle{fancy}
\fancyhf{}
\fancyfoot[C]{\thepage}
\renewcommand{\headrulewidth}{0pt}

\begin{document}

% 封面/头部信息区
\noindent
% Header removed per user request

\vspace{1em}

\begin{center}
    {\Huge\heiti 中国移动专利申请} \\[0.5em]
    {\Huge\heiti 检索报告} \\[1em]
\end{center}

\begin{table}[h]
    \centering
    \renewcommand{\arraystretch}{2}
    \begin{tabular}{|p{3cm}|p{12cm}|}
        \hline
        \textbf{发明名称} & \textcolor{blue}{面向量子计算网络的抗量子攻击节点协同认证系统及方法} \\
        \hline
        \textbf{申报单位} & \textcolor{blue}{研究院} \\
        \hline
        \textbf{检索人} & \textcolor{blue}{许达、xudayj@chinamobile.com、+86-13521894156} \\
        \hline
        \textbf{检索日期} & \colorbox{yellow}{2026.01.12} \\
        \hline
        \textbf{关联项目} & (待填写) \\
        \hline
    \end{tabular}
\end{table}

\vspace{0.5em}

\noindent\fbox{\parbox{0.97\textwidth}{
\textbf{与量子计算的关联说明:}

本发明面向量子计算网络中多节点协同认证的安全需求。随着分布式量子计算、量子云计算的发展,量子计算节点间的安全通信和身份认证面临双重挑战:(1) 必须抵御量子计算机对传统密码的攻击(Shor算法可破解RSA/ECC);(2) 分布式架构要求密钥管理具备容错性。本发明采用NIST标准后量子算法Falcon的门限签名方案,可应用于量子计算集群节点认证、量子-经典混合计算环境安全通道建立、分布式量子计算任务可信调度等场景。
}}

\vspace{1em}

% 正文部分 - 严格按照模板结构

% 二、使用的中文与外文检索关键词
\section*{一、使用的中文与外文检索关键词}

\textbf{中文检索关键词:}
(1) 量子计算网络, 节点认证, 抗量子攻击
(2) Falcon签名, 门限签名, 格密码
(3) 后量子密码学, NTRU格, 多方安全计算
(4) 分布式高斯采样, 量子-经典混合计算

\textbf{外文检索关键词:}
(1) Quantum computing network, node authentication, quantum-resistant
(2) Falcon signature, threshold signature, lattice cryptography
(3) Post-quantum cryptography, NTRU lattice, MPC signature
(4) Distributed Gaussian sampling, quantum-classical hybrid computing

% 三、相关专利文献
\section*{二、相关专利文献}

\renewcommand{\arraystretch}{1.5}
\begin{longtable}{|p{1.5cm}|p{6cm}|p{7.5cm}|}
\hline
\textbf{编号} & \textbf{相关度类别} & \textbf{文献信息 (标题/来源/作者/日期)} \\
\hline
1 & A (基础技术) & \textbf{Falcon: Fast-Fourier Lattice-based Compact Signatures over NTRU} \newline NIST PQC Round 3 Submission, 2020 \newline Fouque, P.A., et al. \\
\hline
2 & A (基础技术) & \textbf{Threshold Cryptosystems From Threshold Fully Homomorphic Encryption} \newline CRYPTO 2018 \newline Boneh, D., et al. \\
\hline
3 & Y (相关技术) & \textbf{Sharing the LUOV: Threshold Post-Quantum Signatures} \newline IMA 2019 \newline Cozzo, D. \& Smart, N.P. \\
\hline
4 & Y (相关技术) & \textbf{Fast Multiparty Threshold ECDSA with Fast Trustless Setup} \newline CCS 2018 \newline Gennaro, R. \& Goldfeder, S. \\
\hline
\end{longtable}

% 四、分析评述
\section*{三、分析评述}

\textbf{1. 与文献1 (Falcon 标准) 的对比分析:}
文献1定义了标准的 Falcon 签名算法,但其设计仅面向单签名者,未涉及多方计算环境下的私钥分片与协作签名。本发明在文献1的基础上,创新性地提出了适用于分布式环境的参数生成与采样方法,解决了标准 Falcon 无法直接应用于门限场景的问题。

\textbf{2. 与文献2 (门限 FHE) 的对比分析:}
文献2提出了基于全同态加密构造门限密码系统的通用框架。虽然提供了理论上的可行性,但全同态加密由于涉及复杂的密文计算,其计算和通信开销巨大,并不适合对签名速度有实时性要求的场景。相比之下,本发明直接针对 Falcon 算法的格结构设计分布式协议,避免了全同态加密的昂贵开销,实现了轻量级、低延迟的量子安全签名。

\textbf{3. 与文献3 (门限 Dilithium/LUOV) 的对比分析:}
文献3虽然涉及了后量子门限签名,但其针对的是 Dilithium 等其他格算法,或者通信开销较大 (O(n))。本发明针对 Falcon 特有的 NTRU 结构,利用 NTT 线性性和 Beaver 三元组预处理,实现了 O(1) 的常数轮次通信,在效率上显著优于文献3的通用方法。

\textbf{4. 与文献4 (门限 ECDSA) 的对比分析:}
文献4是目前主流的门限签名方案,但基于椭圆曲线离散对数问题,不具备抗量子攻击能力。本发明填补了量子安全领域的门限签名空白,提供了与现有系统相当的效率但更高的安全性。

% 五、检索结论
\section*{四、检索结论}

经检索,未发现与本申请全部技术特征相同的现有技术。本申请提出的基于 Falcon 算法的门限签名系统,特别是结合了分布式高斯采样与常数轮次范数验证的方案,具有显著的新颖性和创造性。

\end{document}
