\documentclass[12pt,a4paper]{article}
\usepackage[UTF8]{ctex}
\usepackage{geometry}
\usepackage{amsmath,amssymb}
\usepackage{graphicx}
\usepackage{booktabs}
\usepackage{longtable}
\usepackage{array}
\usepackage{enumitem}
\usepackage{hyperref}
\usepackage{fancyhdr}
\usepackage{xcolor}
\usepackage{colortbl}
\usepackage{setspace}

\geometry{left=2.5cm,right=2.5cm,top=2.5cm,bottom=2.5cm}
\setlength{\headheight}{15pt}
\setstretch{1.5}

% 去掉页眉中的“现有技术检索报告”
\pagestyle{fancy}
\fancyhf{}
\fancyfoot[C]{\thepage}
\renewcommand{\headrulewidth}{0pt}

\begin{document}

% 封面/头部信息区
\noindent
% Header removed per user request

\vspace{1em}

\begin{center}
    {\Huge\heiti 中国移动专利申请} \\[0.5em]
    {\Huge\heiti 检索报告} \\[1em]
\end{center}

\begin{table}[h]
    \centering
    \renewcommand{\arraystretch}{2}
    \begin{tabular}{|p{3cm}|p{12cm}|}
        \hline
        \textbf{发明名称} & \textcolor{blue}{动态多原语密码跳频协议(DMHP)的安全通信系统及方法} \\
        \hline
        \textbf{申报单位} & \textcolor{blue}{研究院} \\
        \hline
        \textbf{检索人} & \textcolor{blue}{许达、xudayj@chinamobile.com、+86-13521894156} \\
        \hline
        \textbf{检索日期} & \colorbox{yellow}{2026.01.26} \\
        \hline
        \textbf{关联项目} & \underline{\hspace{8cm}}(待填写) \\
        \hline
    \end{tabular}
\end{table}

\noindent\fbox{\parbox{0.97\textwidth}{
    \textbf{与量子计算的关联说明:}

本发明面向后量子威胁背景下的长期安全通信需求。量子计算机成熟后可利用Shor算法对RSA/ECC等传统公钥算法造成实质性威胁,同时后量子算法族仍处于持续公开分析演进阶段,存在未来被削弱的可能。本发明提出DMHP动态多原语密码跳频协议,在会话内按时间/序号对不同数学困难问题类别(如格、编码、哈希等)的密码算法与密钥上下文进行跳频,并可选结合多路径传输分散与阈值分片(全息熵分散,HED),以降低“现在存储、未来解密(SNDL)”攻击收益并提升对算法不确定性的韧性。
}}

\vspace{1em}

% 正文部分 - 严格按照模板结构

\section*{一、使用的中文与外文检索关键词}

    \textbf{中文检索关键词:}
(1) 密码跳频, 动态算法切换, 密码敏捷
(2) 后量子密码, 抗量子攻击, SNDL(现在存储未来解密)
(3) 正交安全, 硬问题类别, 算法多样性约束
(4) 多路径传输分散, MPTCP, QUIC多路
(5) 阈值分片, Shamir秘密共享, 纠删码, 全息熵分散

(补充同义/检索友好词:会话内密钥更新/频繁rekey、记录层/包级重密钥、无状态派生、乱序解密、过渡窗口/密钥回滚窗口、密码多样性/多假设安全、信息分散算法IDA)

    \textbf{外文检索关键词:}
(1) cryptographic hopping, algorithm rotation, crypto agility
(2) post-quantum cryptography, quantum-resistant, store now decrypt later
(3) orthogonal security, hard-problem class, algorithm diversity
(4) multipath transport dispersion, MPTCP, multipath QUIC
(5) threshold splitting, secret sharing, erasure coding

(补充同义/检索友好词:intra-session key update, frequent rekeying, per-record rekey, record layer key update, stateless key derivation, out-of-order decryption, key rollover window, overlap window, cryptographic diversity, multi-assumption security, heterogeneous cryptography, information dispersal algorithm (IDA))

\section*{二、检索策略与检索式示例(可复现说明)}

说明:以下为本发明主题的可复现检索策略框架与典型布尔检索式示例,可按实际使用的数据库/平台对字段语法(TI/AB/CL等)做等价替换。

\begin{itemize}[leftmargin=2em]
    \item \textbf{检索对象:}专利文献与非专利文献(标准/草案/论文)。
    \item \textbf{检索字段:}标题(TI)、摘要(AB)、权利要求(CL)及全文(FT)分层检索;优先在TI/AB/CL中做主筛。
    \item \textbf{时间范围:}2000--2026(覆盖后量子迁移与现代传输协议演进阶段)。
    \item \textbf{主题分组:}(G1)会话内算法/密钥动态切换;(G2)无状态派生与乱序容忍;(G3)多算法多样性/多假设;(G4)阈值分片/信息分散;(G5)多路径/路径多样性。
\end{itemize}

\noindent\textbf{英文检索式示例(可组合):}
\begin{itemize}[leftmargin=2em]
    \item (TI/AB/CL: ("crypto agility" OR "algorithm rotation" OR "cipher suite rotation" OR "cryptographic hopping" OR "continuous rekey" OR "frequent rekey" OR "intra-session key update" OR "per-record rekey"))
    \item AND (TI/AB/CL: (packet OR record OR "sequence number" OR "packet number" OR nonce))
    \item (TI/AB/CL: ("stateless key derivation" OR "key derivation" AND ("sequence number" OR "packet number") OR "out-of-order" OR "out of order"))
    \item (TI/AB/CL: ("key rollover" OR "overlap window" OR "grace period" OR "dual decrypt" OR "dual decryption"))
    \item (TI/AB/CL: ("cryptographic diversity" OR "multi-assumption" OR "heterogeneous cryptography" OR "N-version"))
    \item (TI/AB/CL: ("secret sharing" OR threshold OR "information dispersal" OR "IDA" OR "erasure coding"))
    \item (TI/AB/CL: (multipath OR MPTCP OR "multipath QUIC" OR "path diversity" OR "traffic splitting"))
\end{itemize}

\noindent\textbf{中文检索式示例(可组合):}
\begin{itemize}[leftmargin=2em]
    \item (标题/摘要/权利要求:("密码敏捷" OR "算法轮换" OR "动态算法切换" OR "会话内密钥更新" OR "频繁重密钥" OR "记录层密钥更新" OR "按包重密钥" OR "密码跳频"))
    \item AND(标题/摘要/权利要求:(序号 OR 记录号 OR 包号 OR nonce OR "乱序" OR "无状态" OR "派生"))
    \item (标题/摘要/权利要求:("阈值" OR "秘密共享" OR "信息分散" OR "纠删码"))
    \item (标题/摘要/权利要求:("多路径" OR MPTCP OR QUIC OR "路径多样性" OR "分流"))
\end{itemize}

% 三、相关专利文献
\section*{三、相关文献条目}

\renewcommand{\arraystretch}{1.5}
\begin{longtable}{|p{1.5cm}|p{6cm}|p{7.5cm}|}
\hline
\textbf{编号} & \textbf{相关度类别} & \textbf{文献信息 (标题/来源/作者/日期)} \\
\hline
1 & A (相关基础) & \textbf{RFC 8446: The Transport Layer Security (TLS) Protocol Version 1.3} \\ 
& & 来源:IETF Standards Track \quad 作者:Rescorla, E. \quad 日期:2018 \\
& & 证据位置:Section 4.1.1 (Cipher Suite Negotiation) - 现有技术中的会话级协商机制 \\
\hline
5 & A (更贴近:会话内rekey) & \textbf{RFC 8446: TLS 1.3 Key Update mechanism} \\
& & 来源:IETF Standards Track \quad 作者:Rescorla, E. \quad 日期:2018 \\
& & 证据位置:Section 4.6.3 (Key Update) - 会话期间更新流量密钥(但仍为同一套件体制,非算法跳频) \\
\hline
2 & A (相关基础) & \textbf{draft-ietf-tls-hybrid-design: Post-Quantum Hybrid TLS} \\ 
& & 来源:IETF Draft \quad 作者:Stebila, D. et al. \quad 日期:2024 \\
& & 证据位置:Whole Document - 现有技术中混合后量子算法的静态组合方案 \\
\hline
6 & Y (更贴近:QUIC记录/包号) & \textbf{RFC 9001: Using TLS to Secure QUIC} \\
& & 来源:IETF Standards Track \quad 作者:Thomson, M., Turner, S. \quad 日期:2021 \\
& & 证据位置:Packet Number/Key Phase相关章节 - 基于包号与密钥阶段的密钥派生与更新(但不涉及跨困难类别的正交约束) \\
\hline
3 & Y (相关技术) & \textbf{The Spread Spectrum Concept} \\ 
& & 来源:IEEE Transactions on Communications \quad 作者:Scholtz, R.A. \quad 日期:1977 \\
& & 证据位置:Abstract \& Introduction - 频率跳变通信(FHSS)的基本原理(本发明借鉴思想) \\
\hline
7 & Y (更贴近:多路径基础) & \textbf{RFC 8684: Multipath TCP (MPTCP) Protocol} \\
& & 来源:IETF Standards Track \quad 作者:Ford, A. et al. \quad 日期:2020 \\
& & 证据位置:Whole Document - 多路径传输用于可靠性/吞吐(本发明在此基础上引入安全分散联动) \\
\hline
4 & Y (相关技术) & \textbf{Planning for PKI: Best Practices Guide for Deploying Public Key Infrastructure} \\ 
& & 来源:John Wiley \& Sons \quad 作者:Housley, R. \& Polk, T. \quad 日期:2001 \\
& & 证据位置:Chapter on Algorithm Maintenance - 传统的长期算法迁移策略(与本发明动态跳变对比) \\
\hline
8 & Y (更贴近:密码敏捷框架) & \textbf{RFC 7696: Guidelines for Cryptographic Algorithm Agility and Selecting Mandatory-to-Implement Algorithms} \\
& & 来源:IETF Informational \quad 作者:Housley, R., et al. \quad 日期:2015 \\
& & 证据位置:全文(概念与建议)- 讨论算法敏捷的设计原则与MTI算法选择,但未给出会话内细粒度算法跳频/正交约束/无状态派生机制 \\
\hline
\end{longtable}

\vspace{0.5em}

\section*{四、分析评述}

\subsection*{(一)本申请方案的必要技术特征拆分(用于对照)}

为便于与现有技术逐项比对,将本申请方案的关键必要技术特征拆分如下(编号仅用于检索与评述):

\begin{itemize}[leftmargin=2em]
    \item \textbf{F1 会话内动态跳频:}同一会话内,按时间片/序号/记录号等粒度对密码原语(算法/参数族)进行动态切换。
    \item \textbf{F2 确定性无状态派生:}接收端仅依据$\text{SeqID}$(或时间片)与会话种子即可独立派生当前单元的算法索引与密钥材料,容忍乱序/丢包。
    \item \textbf{F3 正交安全约束:}为算法配置困难问题类别元数据,并对相邻单元施加“类别距离$\ge d_{\min}$”之约束,避免在同一困难类别内循环。
    \item \textbf{F4 过渡重叠窗口:}在切换点提供重叠窗口与双解码策略,以提升工程落地可用性。
    \item \textbf{F5 可选HED阈值分片:}对载荷进行$(k,n)$阈值分片/纠删码分散,重构需至少$k$份额。
    \item \textbf{F6 可选MPTD多路径分散:}将不同单元/分片分散至不同物理路径或子流,提高捕获与重构难度。
\end{itemize}

\subsection*{(二)与主要现有技术方案的对比分析}

基于检索到的相关文献(编号1-4),下表总结了现有通用技术路线与本申请方案的对比差异。

\renewcommand{\arraystretch}{1.4}
\begin{longtable}{|p{2.2cm}|p{2.2cm}|p{10.2cm}|}
\hline
	\textbf{对照对象} & \textbf{覆盖特征} & \textbf{差异要点(本申请的技术贡献)}\\
\hline
算法协商/密码敏捷(如文献1、5、8) & 通常覆盖F1(粗粒度) & 现有标准(如TLS 1.3)通常在握手/版本升级阶段确定算法;虽支持Key Update(文献5)实现会话内rekey,但\textbf{仍缺少会话内跨算法原语的细粒度跳频}与F3“正交类别距离”约束;F2无状态派生也未必具备。\\
\hline
后量子标准化/迁移(如文献2) & 可能覆盖“PQC替换/混合” & 往往聚焦“选择某一PQC算法族并替换”,\textbf{缺少跨困难类别的动态轮换与可度量约束(F3)};对SNDL收益削弱通常不以“碎片化跳频+阈值分散”方式实现。\\
\hline
SNDL威胁分析(一般理论) & 提供威胁动机 & 现有研究多为风险分析或建议性结论,\textbf{缺少可执行的会话内跳频与分散机制(F1/F5/F6)}以及工程同步策略(F4)。\\
\hline
多路径传输(如MPTCP等,文献7) & 覆盖F6(传输侧) & 多路径协议的核心目标一般为吞吐/可靠性;\textbf{缺少与算法跳频/阈值分片联动的“算法-路径交叉分散”}整体框架(F1+F5+F6联动)。\\
\hline
\end{longtable}

\subsection*{(三)最接近现有技术、差异特征与技术效果(审查口径化表述建议)}

\textbf{1. 最接近现有技术的确定:}
从工程实现形态看,文献1与文献5(TLS 1.3及其Key Update机制)与文献6(QUIC采用TLS保护并引入基于包号/密钥阶段的派生与更新)均属于"\textbf{在会话期间进行密钥更新/密钥派生}"路线,与本申请的"按序号/时间片驱动的记录/包级保护"较为接近,可作为最接近现有技术族进行对比。

\textbf{2. 相对于最接近现有技术的主要差异特征:}
与上述现有技术相比,本申请至少具有如下差异特征组合(对应前述F1--F4为核心):
\begin{itemize}[leftmargin=2em]
    \item \textbf{差异D1(对应F1):}在同一会话内进行\textbf{跨密码原语/算法族的动态跳频},而不仅是同一算法套件下的密钥更新;
    \item \textbf{差异D2(对应F2):}以$\text{SeqID}$(或时间片)为输入,实现\textbf{确定性无状态派生}得到“算法索引+每单元密钥材料”,从而天然支持丢包/乱序;
    \item \textbf{差异D3(对应F3):}引入困难问题类别(HPC)与类别距离约束$\ge d_{\min}$,实现\textbf{可度量的正交安全轮换规则};
    \item \textbf{差异D4(对应F4):}在切换点提供过渡重叠窗口与双解码策略,以在真实网络抖动/时延条件下保持可用性。
\end{itemize}

\textbf{3. 由差异特征产生的技术问题与技术效果:}
\begin{itemize}[leftmargin=2em]
    \item 通过D1+D3,将连续数据保护的数学假设基础进行跨类别切换,解决"单一算法/同类数学假设被削弱后导致\textbf{连续失守}"的问题,技术效果为:\textbf{降低单点突破的连续影响范围};
    \item 通过D2,在记录/包级场景中无需依赖严格前序状态即可定位算法与密钥上下文,解决“丢包/乱序条件下会话内频繁更新导致\textbf{解密同步困难}”的问题,技术效果为:\textbf{提高乱序容忍与并行处理能力};
    \item 通过D4,解决“切换点时钟漂移/网络抖动导致\textbf{阶段性可用性下降}”的问题,技术效果为:\textbf{在不牺牲核心安全策略的前提下提升工程可用性};
    \item 上述效果共同作用,降低SNDL攻击的"批量采集、未来统一解密"的价值密度,技术效果为:\textbf{提高长期保密的攻击成本与不确定性}。
\end{itemize}

\textbf{4. 非显而易见性说明(组合并非简单拼接):}
即使现有技术已公开"会话内rekey/密钥更新"(文献5)以及"基于包号/阶段的密钥派生与更新"(文献6),本申请的D1--D4组合仍非显而易见,原因在于:
\begin{itemize}[leftmargin=2em]
    \item 将“算法级跳频”(D1)引入记录/包级保护后,必须与无状态派生(D2)和过渡窗口(D4)共同设计,否则在乱序与重传条件下会出现解密失败率上升、重放检测与密钥更新一致性冲突等工程问题;
    \item 现有的“算法敏捷”指导(文献8)多为版本级/策略级治理建议,\textbf{未给出可度量的困难类别距离约束(D3)}以及与记录/包级无状态派生联动的可执行机制;
    \item D3的“类别距离$\ge d_{\min}$”使算法多样性从“可配置”提升为“可度量、可审计、可强制执行”的规则体系,属于机制层面的实质改进,而非简单列举多算法。
\end{itemize}

\subsection*{(四)工程挑战与风险点(审查/落地的防御性说明建议)}

为提升本申请文本在审查阶段的可实施性评价,并为后续说明书实施例补充提供依据,建议在检索报告中明确以下工程挑战与对应的防御性设计要点(不改变核心创造性结论,仅用于“可实施性/可信实现”论证):

\begin{itemize}[leftmargin=2em]
    \item \textbf{E1 不同PQC算法的密文尺寸差异与MTU碎片化风险:}
    由于算法库可能包含格类、编码类、哈希基等不同PQC原语,其密文/封装输出长度存在显著差异。若按包/记录级跳频,输出包长可能随算法切换发生抖动,带来(i)网络层MTU碎片化、额外重传与性能下降;(ii)攻击者通过包长度统计推断当前算法类别的\textbf{侧信道}风险。
    \textbf{防御性建议:}在具体实施方式中引入\textbf{流量整形/自适应填充}模块:设定目标传输单元长度$L_\text{target}$(可取算法库中最大输出长度加冗余量),对短密文追加随机填充,使链路层观测到的包长呈现恒定或伪随机分布,从而同时减轻碎片化与"算法指纹"泄露。

    \item \textbf{E4 接收端状态同步与DoS攻击风险:}
    虽然F2采用无状态派生,但若攻击者注入大量无效大序号包,可能诱发接收端执行高消耗的PQC解密。
    \textbf{防御性建议:}引入\textbf{低成本预过滤机制}:在PQC解密前,基于轻量级对称MAC或Bloom Filter对$\text{SeqID}$的有效性及窗口范围进行预校验,快速丢弃非法流量,保护接收端计算资源。

    \item \textbf{E2 会话主密钥(MasterSecret)的前向安全性(PFS)与回溯风险:}
    若会话期间$\text{MasterSecret}$保持静态,则在攻击者于时刻$T$获取会话主密钥的极端场景下,可能存在对历史数据的回溯解密风险(取决于派生链路与实现细节)。
    \textbf{防御性建议:}在实现例中加入\textbf{棘轮/迭代更新}机制:每隔$N$个受保护单元或每个时间周期,执行$\text{MasterSecret}_{i+1}=\text{KDF}(\text{MasterSecret}_i,\ "ratchet")$并安全擦除旧值。该机制与F2无状态派生可并存:接收端在可配置窗口内保留少量近邻状态(例如最近$w$个棘轮点的种子)用于追赶与乱序容忍。

    \item \textbf{E3 算法库协商、实现体积与启动时延:}
    若算法库覆盖多个困难问题类别,移动端/IoT设备可能面临代码体积、内存与初始化开销。
    \textbf{防御性建议:}可将算法库拆分为"必选最小集(MTI-like)+可选扩展集",握手阶段协商可用集合ID;并允许以"类别级"协商(仅协商HPC类别与版本)降低协商长度,具体算法索引由KDF在集合内部确定。
\end{itemize}

        \textbf{1. 与"算法协商与密码敏捷"相关公开资料的对比:}
传统协议通常在握手阶段确定并锁定单一(或少量)算法套件,在会话期间长时间复用。其弱点在于算法与实现风险集中、容易被SNDL批量采集并等待未来突破。本发明的DLHP在会话内细粒度跳频,降低任一算法长期暴露的价值,且可在不重建会话的前提下完成动态切换。

        \textbf{2. 与“后量子标准化与单算法迁移”相关公开资料的对比:}
单一PQC算法替换可提升抗量子能力,但无法消除算法族在未来被削弱的不确定性。本发明通过“硬问题类别正交约束”,在结构化格、编码理论、哈希基等类别间切换,形成可度量的多样性防线。

        \textbf{3. 与SNDL威胁分析相关公开资料的对比:}
SNDL核心在于长期保存密文等待未来大规模算力/新攻击。本发明通过微碎片化(按包/按块/按分片)减少单一算法保护的连续数据规模,并可选通过阈值分片(HED)将重构门槛提升为“至少解出$k$个份额”。

        \textbf{4. 与多路径传输协议相关公开资料的对比:}
多路径协议通常解决吞吐/可靠性或抗链路故障问题。本发明将多路径与算法跳频结合,形成“空间正交”:攻击者需要同时捕获多条物理路径并跨多个算法类别破译,显著提高完整会话重构难度。

% 五、检索结论
\section*{五、检索结论}

基于当前对包括RFC标准、IETF草案及经典扩频理论等文献的初步检索与分析,结论如下:

\textbf{未发现}单一公开文本同时覆盖本申请的“会话内动态跳频(F1)”、“确定性无状态派生(F2)”、“正交安全约束(F3)”与“过渡重叠窗口(F4)”等核心特征。

现有技术主要存在以下局限:

\begin{itemize}[leftmargin=2em]
    \item 握手阶段算法协商/迁移(密码敏捷),但缺少会话内细粒度跳频与工程可用的过渡机制(F1/F4);
    \item 基于单一PQC算法族的替换或混合,但缺少跨困难类别的“正交安全约束”与可度量的多样性策略(F3);
    \item 多路径传输以性能/可靠性为主,缺少与算法跳频及阈值分片联动的安全分散框架(F6与F1/F5联动)。
\end{itemize}

\textbf{综上所述},本申请提出的DMHP协议在会话内细粒度跳频、正交安全约束及无状态容错机制等方面具有实质性的改进,具备较好的新颖性与创造性前景。进一步地,从工程实现与安全机制耦合角度看:
\begin{itemize}[leftmargin=2em]
    \item \textbf{F2(无状态派生/乱序容忍)与F4(过渡重叠窗口/双解码)并非简单拼接:}其需要在允许乱序和重传的网络条件下同时满足解密成功率、重放检测与密钥更新一致性,属于记录层/包级密钥更新的工程约束体系;
    \item \textbf{F3(困难问题类别距离约束)区别于一般“支持多算法/可配置”的密码敏捷:}本申请提供可度量的“类别距离$\ge d_{\min}$”选择规则,使连续受保护单元跨数学假设类别进行轮换,从而降低同类突破造成连续失守的风险。
\end{itemize}

建议:后续审查中可围绕“会话内频繁rekey/记录层动态切换/阈值分片重构/多路径安全传输结合”等主题进一步扩展检索,以排除相关性较低的组合干扰。

\end{document}
