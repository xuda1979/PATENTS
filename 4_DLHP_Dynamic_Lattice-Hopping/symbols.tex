\documentclass[11pt]{article}
% pdfLaTeX defaults to UTF-8 in modern LaTeX; keep this file UTF-8 encoded.
\usepackage{amsmath, amssymb}
\usepackage{longtable}
\usepackage{geometry}
\usepackage{microtype}
\usepackage[hidelinks]{hyperref}
\geometry{a4paper, margin=1in}

% Consistent formatting for identifiers used across specs/proofs.
\newcommand{\id}[1]{\textsf{#1}}
\newcommand{\field}[1]{\mathrm{#1}}

\date{\today}

\begin{document}

\begin{center}
{\LARGE Mathematical Notation and Symbol Definitions\\[0.2em]Dynamic Multi-Primitive Cryptographic Hopping Protocol (DMP-CHP) (DLHP)\par}
\vspace{0.6em}
{\large \today\par}
\end{center}

\section{List of Symbols}

\renewcommand{\arraystretch}{1.5}
\begin{longtable}{p{0.18\textwidth} p{0.77\textwidth}}
\hline
\textbf{Symbol} & \textbf{Definition} \\
\hline
$\mathcal{M}$ & The space of all possible plaintext messages. \\
$\mathcal{C}$ & The space of all possible ciphertexts. \\
$m$ & A specific plaintext message, $m \in \mathcal{M}$. \\
$S$ & A random variable representing the secret (message). \\
$\id{SeqID}$ & Integrity-protected monotonic sequence identifier for a protected unit (e.g., packet). \\
$\id{GhostSeqID}$ & Monotonic sequence identifier reserved for decoy protected units (stateful counter stored at a node). \\
$n$ & The total number of shares generated in the Holographic Entropy Dispersion (HED) scheme. \\
$k$ & The reconstruction threshold for the HED scheme. \\
$s_i$ & The $i$-th share of the secret, $i \in \{1, \dots, n\}$. \\
$\Lambda$ & The Cryptographic Orthogonality Library, $\Lambda = \{\mathcal{A}_1, \dots, \mathcal{A}_N\}$. \\
$\mathcal{A}_t$ & The active cryptographic algorithm at time step $t$. \\
$\mathcal{H}(\cdot)$ & A cryptographic hash function (modeled as a Random Oracle). \\
$\text{PUF}(c)$ & The response of a Physical Unclonable Function to challenge $c$. \\
$K_{master}$ & Master secret material established during the initial handshake (e.g., from a KEM shared secret). \\
$K_{\field{session}}$ & Session secret used for per-protected-unit derivation and protection (derived from $K_{master}$). \\
$K_{hop}$ & Derived key used specifically for generating or ratcheting the hopping schedule (derived from $K_{\field{session}}$ and optionally device binding such as PUF/TEE). \\
$\text{Hard}(\mathcal{A})$ & The underlying mathematical hard problem of algorithm $\mathcal{A}$. \\
$\text{Class}(\mathcal{A})$ & Hard-problem class label for algorithm $\mathcal{A}$ (e.g., structured lattice, code-based, isogeny-based). \\
$H(X)$ & Shannon entropy of random variable $X$. \\
$H(X|Y)$ & Conditional entropy of $X$ given $Y$. \\
$\theta_{threat}$ & The dynamic parameter vector representing the current threat level. \\
$\mathcal{E}_{env}$ & The environmental entropy collected from network jitter/noise. \\
$\pi^*$ & The optimal policy derived by the Federated Reinforcement Learning agent. \\
$\mathbb{F}_q$ & A finite field of order $q$. \\
\hline
\end{longtable}

\end{document}
