\documentclass[11pt]{article}
% pdfLaTeX defaults to UTF-8 in modern LaTeX; keep this file UTF-8 encoded.
\usepackage{amsmath, amssymb, amsthm, mathtools}
\usepackage{geometry}
\usepackage{microtype}
\usepackage[hidelinks]{hyperref}
\usepackage{bookmark}
\geometry{a4paper, margin=1in}

\author{Technical Specification Supplement}
\date{\today}

\hypersetup{
    pdftitle={Mathematical Security Proofs for DMP-CHP (DLHP)},
    pdfauthor={Technical Specification Supplement},
    pdfsubject={Security proofs and definitions for DMP-CHP / DLHP}
}

\newtheorem{theorem}{Theorem}
\newtheorem{lemma}[theorem]{Lemma}
\newtheorem{definition}{Definition}
\newtheorem{proposition}[theorem]{Proposition}

% Consistent identifier formatting across docs.
\newcommand{\id}[1]{\textsf{#1}}
\newcommand{\field}[1]{\mathrm{#1}}

\begin{document}

\begin{center}
{\LARGE Mathematical Security Proofs for\\[0.2em]Dynamic Multi-Primitive Cryptographic Hopping Protocol (DMP-CHP) (also referred to as DLHP)\par}
\vspace{0.4em}
{\large Technical Specification Supplement\par}
\vspace{0.6em}
{\large \today\par}
\end{center}

\begin{abstract}
This document provides formal mathematical arguments supporting the security goals of the Dynamic Multi-Primitive Cryptographic Hopping Protocol (DMP-CHP), also referred to herein as DLHP, focusing on (i) Holographic Entropy Dispersion (HED), (ii) orthogonal security under poly-algorithmic encryption, and (iii) unpredictability of the hopping schedule.
\end{abstract}

\section{Holographic Entropy Dispersion (HED)}

\subsection{Preliminaries}
Let $\mathcal{M}$ be the message space and $\mathcal{S}$ be the share space. The system utilizes a $(k, n)$ threshold secret sharing scheme (e.g., Shamir's Secret Sharing over a finite field $\mathbb{F}_q$).

\begin{definition}[Holographic Fragmentation]
A message $m \in \mathcal{M}$ is divided into $n$ shares $\{s_1, s_2, \dots, s_n\}$ such that:
\begin{enumerate}
    \item \textbf{Reconstruction:} Any subset of $k$ shares can reconstruct $m$.
    \item \textbf{Secrecy:} Any subset of fewer than $k$ shares reveals no information about $m$.
\end{enumerate}
Each share $s_i$ is subsequently encrypted using a distinct cryptographic algorithm $\mathcal{E}_i \in \Lambda$, where $\Lambda$ is the Orthogonality Library.
\end{definition}

\subsection{Information-Theoretic Security Proof}

\begin{theorem}[Perfect Secrecy of Sub-Threshold Fragments]
Let $S$ be the random variable representing the secret message and let $V_T = \{s_{i_1}, s_{i_2}, \dots, s_{i_t}\}$ be a set of $t < k$ shares intercepted by an adversary. Then:
\[ H(S \mid V_T) = H(S) \]
where $H(\cdot)$ denotes Shannon entropy.
\end{theorem}

\begin{proof}
Consider Shamir's scheme where the secret $S = a_0$ is the constant term of a random polynomial $P(x) = a_0 + a_1x + \dots + a_{k-1}x^{k-1}$ of degree $k-1$ over $\mathbb{F}_q$, with coefficients $a_1, \dots, a_{k-1}$ chosen uniformly at random.

A set of $t$ shares corresponds to $t$ points $(x_j, y_j)$ where $y_j = P(x_j)$.
For any candidate secret $s' \in \mathbb{F}_q$ and any set of $t < k$ shares, there exists a unique polynomial $P'(x)$ of degree $k-1$ such that $P'(0) = s'$ and $P'(x_j) = y_j$ for all $j \in \{1, \dots, t\}$.

Since the remaining $k-1-t$ coefficients are free variables, there are exactly $q^{k-1-t}$ polynomials consistent with the shares and the candidate secret. This count is independent of the value of $s'$.
Therefore, for any observation of $t < k$ shares, every possible secret $s'$ is equally likely.

\[ \Pr[S = s | V_T = v] = \Pr[S = s] \]
\[ H(S \mid V_T) = H(S) \]
Thus, the shares provide zero mutual information about the secret $S$.
\end{proof}

\section{Orthogonal Security Analysis}

\subsection{Assumptions}
Let $\Lambda = \{\mathcal{A}_1, \mathcal{A}_2, \dots, \mathcal{A}_N\}$ be a set of cryptographic algorithms.
We define $Hard(\mathcal{A}_i)$ as the underlying mathematical problem for algorithm $\mathcal{A}_i$ (e.g., SIS, LWE, MQ, Code-Decoding).

\begin{definition}[$\epsilon$-Orthogonality (advantage form)]
Two algorithms $\mathcal{A}_i$ and $\mathcal{A}_j$ are $\epsilon$-orthogonal if access to efficient auxiliary information or solver capabilities for $Hard(\mathcal{A}_i)$ does not increase a polynomial-time adversary's advantage in solving $Hard(\mathcal{A}_j)$ by more than $\epsilon$.
Formally, for any PPT adversary $\mathcal{B}$,
\[ \text{Adv}_{\mathcal{B}}(Hard(\mathcal{A}_j) \mid \text{Aux}_{\mathcal{A}_i}) \leq \text{Adv}_{\mathcal{B}}(Hard(\mathcal{A}_j)) + \epsilon, \]
where $\epsilon$ is negligible in the security parameter and \(\text{Aux}_{\mathcal{A}_i}\) denotes auxiliary information derivable from running efficient solvers or side-information about $Hard(\mathcal{A}_i)$.
\end{definition}

\subsection{Joint Security of Holographic Session}

\begin{theorem}[Composite Hardness]
For a $(k, n)$ HED session, let $E_i$ be the event that an adversary successfully breaks algorithm $\mathcal{A}_i$. Assuming $\epsilon$-orthogonality between all pairs in $\Lambda$, the probability of compromising the payload $P_{compromise}$ is:
\[ P_{\field{compromise}} \leq \sum_{\substack{T \subseteq \{1,\dots,n\}\\ |T|=k}} \left( \prod_{j \in T} \Pr[E_j] \right) + \delta \]
where $\delta$ is a negligible term representing higher-order correlations.
\end{theorem}

\begin{proof}
To recover the payload, the adversary must obtain $k$ valid shares.
Let $S_i$ be the share protected by $\mathcal{A}_i$. Recovering $S_i$ requires event $E_i$.
The adversary succeeds if they break any subset of $k$ algorithms.

Under the assumption of pairwise $\epsilon$-orthogonality (as defined above), the dependence between events $E_i$ and $E_j$ is bounded: knowledge or solver capabilities for one gives at most negligible extra advantage for the other, up to $\epsilon$ terms. We therefore treat the joint success probability for a chosen subset $T$ of size $k$ as approximately the product of individual success probabilities, with a small residual term capturing higher-order correlations.

Let $p_i = \Pr[E_i]$. Then, for any subset $T$ of size $k$ an upper bound on the adversary's success probability is:
\[ \Pr\Big[\bigcap_{j \in T} E_j\Big] \leq \prod_{j \in T} p_j + \delta_T, \]
where $\delta_T$ captures higher-order correlations and is expected negligible when $\epsilon$ is negligible. Summing over all $\binom{n}{k}$ subsets yields the composite bound in the theorem statement, with the global residual denoted $\delta$.

If $p_i \approx 2^{-\lambda}$ for all $i$, the dominant term scales like $2^{-k\lambda}$, i.e., the effective security parameter grows approximately linearly with $k$ under the independence approximation.

The bound is conservative: if algorithms share substantial structure (e.g., all lattice-based with correlated parameters), $\epsilon$ and $\delta$ may be non-negligible, and the product approximation would no longer hold. Thus orthogonality and diversity in hard problems are essential to the claimed scaling.
\end{proof}

\section{Hopping Schedule Unpredictability}

\subsection{Schedule Generation}
The schedule is generated via a function $F(K, t, \eta)$, where $K$ is a master secret (e.g., $K_{\field{session}}$), $t$ is the time/index (or packet sequence identifier), and $\eta$ is auxiliary entropy.

\begin{definition}[PUF-Bound PRF]
Let $\text{PUF}(c)$ be a physical unclonable function response to challenge $c$.
The hopping key is $K_{\field{hop}} = \text{HKDF}(K_{\field{session}}, \text{PUF}(\id{nonce}))$.
The schedule at step $i$ is $A_i = \mathcal{H}(K_{\field{hop}} \parallel i) \bmod |\Lambda|$.
\end{definition}

\begin{theorem}[Future Unpredictability]
If $\text{PUF}(\cdot)$ has min-entropy $\gamma$ and $\mathcal{H}$ is modeled as a Random Oracle, then for an adversary $\mathcal{B}$ with view of previous algorithms $\{A_0, \dots, A_{t-1}\}$ but without physical access to the device:
\[ \Pr[\mathcal{B} \text{ predicts } A_t] \leq \frac{1}{|\Lambda|} + \text{negl}(\lambda) \]
\end{theorem}

\begin{proof}
The output $A_i$ depends on $K_{\field{hop}}$. Access to $\{A_0, \dots, A_{t-1}\}$ gives information about $K_{\field{hop}}$ only if $\mathcal{H}$ can be inverted.
Even if $K_{\field{session}}$ is compromised (e.g., memory dump), $K_{\field{hop}}$ remains unknown because $\text{PUF}(\id{nonce})$ cannot be computed without the physical hardware instance.
Assuming the PUF response has sufficient entropy $\gamma$, the conditional entropy $H(K_{\field{hop}} \mid K_{\field{session}}) \approx \gamma$.
Therefore, the sequence $\{A_i\}$ remains pseudo-random to any observer lacking the physical device.
\end{proof}

\section{Federated Mutation Function}

The schedule mutation rule is defined as:
\[ S_{t+1} = \mathcal{G}(S_t, \theta_{threat}, \mathcal{E}_{env}) \]
where $\mathcal{G}$ is the evolution function derived from the Federated Reinforcement Learning agent.

Let the loss function for the RL agent be:
\[ \mathcal{L}(\pi) = \mathbb{E}_{\tau \sim \pi} \left[ \sum_{t=0}^T \gamma^t (R_{security}(s_t, a_t) - \lambda C_{bandwidth}(s_t, a_t)) \right] \]
By optimizing this objective across $M$ nodes without sharing raw trajectories $\tau$, the system converges to a global policy $\pi^*$ that maximizes security while minimizing bandwidth overhead.

\end{document}
